\documentclass[11pt,fleqn,twoside]{article}
\usepackage{makeidx}
\makeindex
\usepackage{palatino} %or {times} etc
\usepackage{plain} %bibliography style
\usepackage{amsmath} %math fonts - just in case
\usepackage{amsfonts} %math fonts
\usepackage{amssymb} %math fonts
\usepackage{lastpage} %for footer page numbers
\usepackage{fancyhdr} %header and footer package
\usepackage{mmpv2}
\usepackage{url}

% the following packages are used for citations - You only need to include one.
%
% Use the cite package if you are using the numeric style (e.g. IEEEannot).
% Use the natbib package if you are using the author-date style (e.g. authordate2annot).
% Only use one of these and comment out the other one.
\usepackage{cite}
%\usepackage{natbib}

\begin{document}

\name{Daniel Atkinson}
\userid{daa9}
\projecttitle{Arduino based obstacle avoidance robot}
\projecttitlememoir{Obstacle avoidance robot} %same as the project title or abridged version for page header
\reporttitle{Progress Report}
\version{0.1}
\docstatus{Draft}
\modulecode{CS39440}
\supervisor{Dave Barnes} % e.g. Neil Taylor
\supervisorid{abc}
\wordcount{}

%optional - comment out next line to use current date for the document
%\documentdate{26th October 2011}
\mmp

\setcounter{tocdepth}{3} %set required number of level in table of contents
\tableofcontents

\newpage

%==============================================================================
\section{Project Summary}
%==============================================================================
The aim of the project is to build and program an autonomous robot that can move around freely within an environment and avoid coliding with obstacles it may find.
\\I tend to experiment with electronic components on a regular basis.  Due to this I try and make small systems and get them working, originaly nothing more than a simple timer or an audio amplifier.  Naturaly the progression from this would be to move onto microcontrollers.
\\

%==============================================================================
\section{Current Progress}
%==============================================================================


%==============================================================================
\section{Planning}
%==============================================================================
Text in here.


\nocite{*} % include everything from the bibliography, irrespective of whether it has been referenced.

% the following line is included so that the bibliography is also shown in the table of contents. There is the possibility that this is added to the previous page for the bibliography. To address this, a newline is added so that it appears on the first page for the bibliography.
\newpage
\addcontentsline{toc}{section}{Annotated Bibliography}

%
% example of including an annotated bibliography. The current style is an author date one. If you want to change, comment out the line and uncomment the subsequent line. You should also modify the packages included at the top (see the notes earlier in the file) and then trash your aux files and re-run.
%\bibliographystyle{authordate2annot}
\bibliographystyle{IEEEannot}
\renewcommand{\refname}{Annotated Bibliography}  % if you put text into the final {} on this line, you will get an extra title, e.g. References. This isn't necessary for the outline project specification.
\bibliography{mmp} % References file


\end{document}
