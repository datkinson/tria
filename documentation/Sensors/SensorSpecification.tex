% Type of the document is an article
\documentclass{article}

% use a graphics package for including images
\usepackage{graphicx}

% begin the document
\begin{document}

% give the document a title
\title{Sensor Specification}

% list the author of the document
\author{Daniel Atkinson}

% print the title, author, etc... here
\maketitle

% start the abstract
\begin{abstract}

Details of various sensors that could be used to detect nearby objects and justification for which is chosen.
\\The main purpose is to focus towards reactive systems as they are faster and should enable to robot to stay in motion for longer periods of time, as opposed to sitting still and scanning its environment before deciding what to do.

%end the abstract
\end{abstract}


\section{Advantages and disadvantages}
There are several different types of sensors that can be used to determine whether or not there is an object in proximity.

\subsection{Ultrasound}
Ultrasound often called sonar uses sound to measure distance.  They are typicaly very accurate in the right conditions, but do suffer from several shortcomings.
\\Different materials can change the reading such as a sponge that will absorb some of the sound.  The only major issue really is the 'ghost echo'.  As the sound can bounce off of multiple surfaces it can echo, where the same pulse is heard many times.  Due to the way sonar works by judging distance via how long it takes for the emitted pulse to bounce back, if the surface was at an angle it can bouce away from the sonar into other objects and continue to bounce around and possibly get back to the sonar after multiple reflections.  This of course takes longr and can give a false distance reading based on this extended time making the device think an object is further away than it actually is.
\subsection{Infrared}
There are two types of infrared sensors.  The first type is binary and indicated if there is an object in proximity, but with no more precision than that.
\\The second type is analogue, this types output will be multiple bit.  An analogue output is good for determining the sensors distance from an object in proximity.
\\The main downside of infrared is direct sunlight or just ambient light.  As the sensor emits infrared and measures how much come back, extra infrared sources can disrupt the reading giving inaccurate values.
\\The Sharpe IR Range Finder modules I have found which seem to be the most suitable for this type of project due to the fact that they appear to work well even in well lit areas.  These sensors work by triangulation.  They emit a pulse of IR and measures the angle that it comes back at (if it comes back at all), this is what determines the distance from the object.
\\Another downside of this is that it can produce different reading for objects that are the same distance away based on the material or even the colour of the reflecting surface.

\subsection{Camera}


\section{Comparison based on project needs}

\subsection{Accuracy}

\subsection{Power}

\subsection{Processing}

% insert an image
%\begin{figure}
%\centering
%    \includegraphics[width=3.0in] {figures/example.png}
%    \caption{example image}
%    \label{example figure}
%\end{figure}


% end the document
\end{document}
