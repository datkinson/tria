% Type of the document is an article
\documentclass{article}

% use a graphics package for including images
\usepackage{graphicx}

% begin the document
\begin{document}

% give the document a title
\title{Motor Specification}

% list the author of the document
\author{Daniel Atkinson}

% print the title, author, etc... here
\maketitle

% start the abstract
\begin{abstract}

Details of various motor types to be used for the drive system and the configuration they will be used in.

%end the abstract
\end{abstract}


\section{Advantages and disadvantages}
There are several types of motors that can be used to drive wheels.  Each type will have various advantages and disadvantages over the other types.  This section will detail these and is aimed to come to a conclusion as to which type will be best for the application of Tria.

\subsection{DC motors}
DC motors as the name describes use direct current to drive them.  When power is supplied to one side and the other is connected to ground the motor is on.  When the curent is revesed by switching which side has power and which side is connected to ground the motor is also on, but turning in the opposite direction.  When neither side is supplied power it is off, and when both sides are supplied with power that is also off but potentialy damaging.

[ Figure to go here of a condition diagram of motor states ]

\subsection{Servos}

\subsection{Stepper motors}


\section{Comparison based on project needs}

The project requires precise control over the drive system.  This is to try and be as accurate as possible as tot he distance the robot has moved and turned.
\\This is not going to be perfect or even close to perfect, but in my eyes, any attempt to reduce the amount of error carried forward the better.
\subsection{Accuracy}

\subsection{Power}

% insert an image
%\begin{figure}
%\centering
%    \includegraphics[width=3.0in] {figures/example.png}
%    \caption{example image}
%    \label{example figure}
%\end{figure}


% end the document
\end{document}
