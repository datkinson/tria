\documentclass[11pt,fleqn,twoside]{article}
\usepackage{makeidx}
\makeindex
\usepackage{palatino} %or {times} etc
\usepackage{plain} %bibliography style 
\usepackage{amsmath} %math fonts - just in case
\usepackage{amsfonts} %math fonts
\usepackage{amssymb} %math fonts
\usepackage{lastpage} %for footer page numbers
\usepackage{fancyhdr} %header and footer package
\usepackage{mmp} 
\usepackage{url}
\usepackage{cite}

\begin{document}

\name{Daniel Atkinson}
\userid{daa9}
\projecttitle{Arduino based obstacle avoidance robot}
\projecttitlememoir{Arduino based obstacle avoidance robot} %same as the project title or abridged version for page header
\reporttitle{Outline Project Specification}
\version{0.1}
\docstatus{Draft}
\modulecode{CS39440}
\supervisor{Dave Barnes} % e.g. Neil Taylor
\supervisorid{abc}

%optional - comment out next line to use current date for the document
%\documentdate{19th October 2012} 
\mmp

\setcounter{tocdepth}{3} %set required number of level in table of contents


%==============================================================================
\section{Project description}
%==============================================================================
When making an autonomous robot, regardless of its primary function, it will need to be able to avoid damaging itself on its environment or somehow restricting itself.  This can be in the form of a reactive system or something more inteligent.  To not get too close to an object in the environment or even in some cases not colliding with itself, such as in the case of an arm.
\\The aim of the project is to design and build a wheeled robot that can independantly (with no user input) move around within an environment and not get hindered by any obstacles that may be in its path.
\\This will be done by building a bump skirt out of sonar or infra red sensors to avoid objects in direct proximity.  These will be placed all around the robot.  The other method will be some more of these sensors mounting in such a way that they can move to look around the environment and attempt to plan the next direction to move and possibly a route.
%==============================================================================
\section{Work to be tackled}
%==============================================================================
\subsection{Hardware}
Some of the work to be done will be designing the hardware of the system.  This would have to take into consideration things like battery life expected for a 'decent' runtime, overall weight so that it is not over encumbered, size to make it practical and speed of operation as to not take too long and become useless or a hinderance to its purpose.
\\The various pieces of hardware or even the various subsystems will need to be bale to talk to each other reliably.
\subsection{Software}
Software is the major component to this project where all of the sub-systems will be independant entities that do their own job and comminicate with the other systems to form a complete system.  These various systems may be running on seperate pieces of hardware and even writen in different languages, this must also be taken into account.


%==============================================================================
\section{Project deliverables}
%==============================================================================


%==============================================================================
\section{Initial bibliography}
%==============================================================================

This template uses BibTeX in a LaTeX document to manage the bibliography. To include citations, you 
can use the \textbackslash cite command, for example \cite{NumericalRecipes}\cite{MarksPaper}\cite{FailBlog}\cite{kittenpic_ref}.  The bibliographic tools are not a requirement in this document, but you are welcome to use them.  

% example of including
\bibliographystyle{plain}
\renewcommand{\refname}{}  % if you put text into the final {} on this line, you will get an extra title, e.g. References. This isn't necessary for the outline project specification. 
\bibliography{mmp} % References file


\end{document}
